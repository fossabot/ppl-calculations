\documentclass[12pt]{article}
\usepackage[a4paper, margin=0.8in]{geometry}
\usepackage{graphicx}
\usepackage{float}
\usepackage{caption}
\usepackage[sfdefault]{roboto}
\usepackage{fancyhdr}
\usepackage{titlesec}
\usepackage{colortbl}
\usepackage[table,xcdraw]{xcolor}

\titleformat{\section}
  {\normalfont\Large\bfseries\filcenter} % Center-align
  {}
  {0pt}
  {}

\pagestyle{fancy}
\fancyhf{}
\fancyhead[L]{\textbf{ {{.CallSign}} }}
\fancyhead[C]{\textbf{ {{.Reference}} }}
\fancyhead[R]{\textbf{ {{.Generated}} }}
\cfoot{\textbf{\thepage}}

\setlength{\headheight}{16pt}
\renewcommand{\sectionmark}[1]{\markboth{#1}{}}

\begin{document}

\section*{Gewicht en Balans}

{{ if not .WeightAndBalanceTakeOff.WithinLimits }}
{\small
\noindent
\colorbox{red!80}{%
    \parbox{\textwidth}{%
        \centering
        {\textcolor{white}{\textbf{De huidige gewichts- en balansberekening geeft aan dat de belading van het vliegtuig buiten de
        toegestane limieten valt. Controleer en herbereken de gewichts- en balansverdeling zorgvuldig om te
        voldoen aan de veiligheidsvoorschriften.}}}%
    }%
}
}
{{ end }}

\begin{figure}[H]
    \centering
    \includegraphics[width=0.6\textwidth]{wb.png}
\end{figure}

{\small
\begin{table}[H]
    \centering
    \renewcommand{\arraystretch}{1.2}
    \setlength{\tabcolsep}{5pt}
    \caption*{\textbf{Take-off}}
    \begin{tabular}{|l|c|c|c|}
        \hline
        \rowcolor[HTML]{AAAAAA}
        \textbf{NAME} & \textbf{MASS [KG]} & \textbf{ARM [M]} & \textbf{MASS MOMENT [KG M]} \\ \hline
        {{ range $index, $item := .WeightAndBalanceTakeOff.Items }}
            {{ $item.Name }} & {{ $item.LeverArm }} & {{ $item.Mass }} & {{ $item.MassMoment }} \\ \hline
        {{ end }}
        \rowcolor[HTML]{AAAAAA}
        \textbf{TOTAL} & \textbf{ {{ .WeightAndBalanceTakeOff.Total.LeverArm }} } & \textbf{ {{ .WeightAndBalanceTakeOff.Total.Mass }} } & \textbf{ {{ .WeightAndBalanceTakeOff.Total.MassMoment }} } \\ \hline
    \end{tabular}
\end{table}

\begin{table}[H]
    \centering
    \renewcommand{\arraystretch}{1.2}
    \setlength{\tabcolsep}{5pt}
    \caption*{\textbf{Landing}}
    \begin{tabular}{|l|c|c|c|}
        \hline
        \rowcolor[HTML]{AAAAAA}
        \textbf{NAME} & \textbf{MASS [KG]} & \textbf{ARM [M]} & \textbf{MASS MOMENT [KG M]} \\ \hline
        {{ range $index, $item := .WeightAndBalanceLanding.Items }}
            {{ $item.Name }} & {{ $item.LeverArm }} & {{ $item.Mass }} & {{ $item.MassMoment }} \\ \hline
        {{ end }}
        \rowcolor[HTML]{AAAAAA}
        \textbf{TOTAL} & \textbf{ {{ .WeightAndBalanceLanding.Total.LeverArm }} } & \textbf{ {{ .WeightAndBalanceLanding.Total.Mass }} } & \textbf{ {{ .WeightAndBalanceLanding.Total.MassMoment }} } \\ \hline
    \end{tabular}
\end{table}
}

\newpage
\section*{Brandstofplanning}


{{ if not .FuelSufficient }}
{\small
\noindent
\colorbox{red!80}{%
\parbox{\textwidth}{%
\centering
{\textcolor{white}{\textbf{
De huidige brandstofvoorraad van {{ .FuelTotal }} is onvoldoende om de geplande vlucht veilig uit te
voeren. Er moet minimaal {{ .FuelExtraAbs }} extra brandstof worden bijgetankt om te voldoen
aan de veiligheidsvoorschriften.
}}}%
}%
}
}
{{ end }}

{\small
\begin{table}[H]
    \centering
    \renewcommand{\arraystretch}{1.5}
    \setlength{\tabcolsep}{10pt}
    \begin{tabular}{|l|c|}
        \hline
        \rowcolor[HTML]{AAAAAA}
        \textbf{Branstofcategorie} & \textbf{Hoeveelheid} \\ \hline
        Taxi Brandstof & {{ .FuelTaxi }}           \\ \hline
        Reisbrandstof (17L/H) & {{ .FuelTrip }}           \\ \hline
        Onvoorziene brandstof (10\%) & {{ .FuelContingency }}           \\ \hline
        Brandstof alternatieve luchthaven & {{ .FuelAlternate }}           \\ \hline
        Eindreservebrandstof (45 minuten) & {{ .FuelReserve }}           \\ \hline
        Extra brandstof & {{ .FuelExtra }}           \\ \hline
        \rowcolor[HTML]{AAAAAA}
        \textbf{Totaal} & \textbf{ {{ .FuelTotal }} } \\ \hline
    \end{tabular}
\end{table}
}

\section*{Prestaties}

{{ if not .WeightAndBalanceTakeOff.WithinLimits }}
{\small
\noindent
\colorbox{red!80}{%
    \parbox{\textwidth}{%
        \centering
        {\textcolor{white}{\textbf{
De prestaties kunnen niet worden berekend omdat de huidige gewichts- en balansberekening aangeeft dat de belading van het vliegtuig buiten de toegestane limieten valt. Controleer en herbereken de gewichts- en balansverdeling zorgvuldig om te voldoen aan de veiligheidsvoorschriften
}}}%
    }%
}
}
{{ else }}
{\small
\begin{table}[H]
    \centering
    \renewcommand{\arraystretch}{1.5}
    \setlength{\tabcolsep}{10pt}
    \begin{tabular}{|l|c|}
        \hline
        \rowcolor[HTML]{AAAAAA}
        \textbf{Name} & \textbf{Distance [m]} \\ \hline
        Take-off Run Required (Ground Roll) & {{ .TakeOffRunRequired }} \\ \hline
        Take-off Distance Required & {{ .TakeOffDistanceRequired }} \\ \hline
        Landing Distance Required & {{ .LandingDistanceRequired }} \\ \hline
        Landing Ground Roll Required & {{ .LandingGroundRollRequired }} \\ \hline
    \end{tabular}
\end{table}
}

\newpage
\begin{figure}[H]
    \centering
    \includegraphics[width=1\textwidth]{tdr.png}
\end{figure}

\begin{figure}[H]
    \centering
    \includegraphics[width=1\textwidth]{ldr.png}
\end{figure}
{{ end }}

\end{document}
